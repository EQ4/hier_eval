% -----------------------------------------------
% Template for ISMIR 2014
% (based on earlier ISMIR templates)
% -----------------------------------------------

\documentclass{article}
\usepackage{ismir2014,amsmath,cite}
\usepackage{graphicx}
\usepackage{amsfonts}
\usepackage{brian}

% Title.
% ------
\title{Hierarchical evaluation of music segment boundary detection}

% Single address
% To use with only one author or several with the same address
% ---------------
%\oneauthor
% {Names should be omitted for double-blind reviewing}
% {Affiliations should be omitted for double-blind reviewing}

% Two addresses
% --------------
\twoauthors
  {First author} {School \\ Department}
  {Second author} {Company \\ Address}

% Three addresses
% --------------
%\threeauthors
  %{First author} {Affiliation1 \\ {\tt author1@ismir.edu}}
  %{Second author} {\bf Retain these fake authors in\\\bf submission to preserve the formatting}
  %{Third author} {Affiliation3 \\ {\tt author3@ismir.edu}}

% Four addresses
% --------------
%\fourauthors
%  {First author} {Affiliation1 \\ {\tt author1@ismir.edu}}
%  {Second author}{Affiliation2 \\ {\tt author2@ismir.edu}}
%  {Third author} {Affiliation3 \\ {\tt author3@ismir.edu}}
%  {Fourth author} {Affiliation4 \\ {\tt author4@ismir.edu}}

\begin{document}
%
\maketitle
%
\begin{abstract}
Structure in music is traditionally analyzed hierarchically: large scale sections at the highest level can be sub-divided and refined down to the short melodic ideas at the motivic level. 
However, typical algorithmic approaches to structural annotation produce flat temporal partitions of a track, which are commonly evaluated against a similarly flat, human-produced
annotation, effectively discarding all notions of structural depth in the annotation.
Recently, collections of hierarchical structure annotations have been published, but at present, no evaluation techniques exist to evaluate against these rich structural annotations.
We propose a method to evaluate structural boundary detection in the context of hierarchical annotation.
The proposed method transforms boundary detection into a ranking problem, and facilitates the comparison of flat and hierarchical annotations.
We demonstrate the proposed method with various synthetic and real world examples. 
\end{abstract}
%
\section{Introduction}\label{sec:introduction}

  Segment Hierarchy Agreement Grade (SHAG)
Segmentation is hierarchical.
SALAMI: dataset with hierarchical annotations.
Rank retrieval in order to evaluate hierarchical annotations.
Organization of the paper.

\section{Segment Hierarchy Agreement Grade}\label{sec:eval_desc}

\subsection{Music segmentation background}

Let $X \in \mathbb{R}^{d\times n}$ denote a song as represented by a time series of 
feature data, where $d$ denotes the dimension of features, and $n$ denotes the
time duration at some fixed resolution $f_r$ (\eg, 10Hz).
Existing approaches to boundary evaluation pre-suppose a single partitioning into $m$
temporally contiguous regions, so that $X=[X_1|X_2|\cdots|X_m]$.  An estimated
partitioning can then be evaluated by comparing the estimated boundaries between
regions to those in a reference annotation.

Instead, we will suppose that $X$ can be hierarchically decomposed into a tree structure.
In this view, we have segmentations at multiple resolutions or levels of granularity.
The question now is how to evaluate against tree-structured segmentation.

\subsection{Notation}

Let $S \in \mathcal{P}(X)$ denote a temporally contiguous partition of the columns of
$X$.
We will use the subscripts $S_R$ and $S_E$ to denote \emph{reference} and \emph{estimation}, respectively.
$S(i)$ will denote the identifier of the partition containing the $i$th frame (column $X_i$).

Hierarchical segmentations will be denoted by $H$ and follow the subscript conventions
above.  $H(i)$ will denote the most specific segment containing $i$, that is, the
segment containing $i$ at the deepest level in the hierarchy.
$H(i, j)$ will denote the least common ancestor of $H(i)$ and $H(j)$.
We will denote precedence (containment) of segments by $\prec$ and $\preceq$: \eg,
$H(i) \preceq H(i, j)$. 

Note that flat segmentations are a special case of hierarchical segmentations, where there is one node at the root of the hierarchy containing all frames, and then $m$ segments at the next level down which partition $X$.

Restricting to a query frame $q$, $H(q, \cdot)$ induces a partial ranking over the
remaining frames.  Frames contained in $H(q)$ are considered maximally relevant,
followed by those in $H(q)$'s immediate ancestor, and so on.
This observation will be the key to our evaluation, as it provides a connection
between hierarchical time-series decompositions and ranking evaluation.


\subsection{Reduction to ranking}


We can reduce segmentation evaluation to a ranking evaluation problem as follows.
Let $q$ denote an arbitrary frame, and let $i$ and $j$ denote any two frames such that $S_R(q) = S_R(i)$ and $S_R(q) \neq S_R(j)$.
In this case, $i$ may be considered \emph{relevant} for $q$, and $j$ is considered \emph{irrelevant} for $q$.
This leads to a straightforward reduction to bipartite ranking.

Assuming that both $S_E$ and $S_R$ have $n$ frames, we can formally describe a frame
recall metric:
\begin{equation}
f(q ; S_E) \defeq 
\sum_{\substack{i \in S_R(q)\\ j \notin S_R(q)}}
\frac{\ind{S_E(q) = S_E(i) \neq S_E(j) }}{|S_R(q)|\cdot (n -
|S_R(q)|)}.\label{flatrecall}
\end{equation}
That is, the score for frame $q$ is the fraction of pairs $(i, j)$ for which $S_E$
agrees with $S_R$ with respect to $q$ (\ie, membership in the segment).

Averaging over all frames $q$ yields an average frame recall score:
\begin{equation}
\rho(S_E) \defeq \frac{1}{n} \sum_q f(q ; S_E).\label{avgrecall}
\end{equation}
% TODO:   2014-05-05 00:38:15 by Brian McFee <brm2132@columbia.edu>
% talk about why we don't need to consider precision here
% predictions are mutually exclusive
% the usual way to cheat at recall is to make too many predictions..
% but to do that for q1, you'd have to make fewer predictions for q2,
\nocite{levy2008structural}


\subsection{Partial Ranking}

\Cref{flatrecall} is defined in terms of segment membership (in)equalities, but 
we can equivalently express the function using strict precedences in a (flat)
segment hierarchy.
Rather than compare $i$ and $j$ where $S(q) = S(i) \neq S(j)$, we can express this as $H(q, i) \prec H(q, j)$.
That is, $q$ and $i$ merge before $q$ and $j$:
\begin{equation}
g(q ; H_E) \defeq \sum_{\substack{(i, j)\\ H_R(q, i) \prec H_R(q, j)}}
\frac{\ind{H_E(q, i) \prec H_E(q, j)}}{Z_q},\label{hierrecall}
\end{equation}
where $Z_q$ is a normalization term that counts the number of terms in the summation.

Just as in \cref{flatrecall}, $g$ can be viewed as a classification accuracy of
correctly predicting pairs $(i, j)$ as positive ($q$ and $i$ merge first) or negative
($q$ and $j$ merge first).  The case where $H(q, i) = H(q, j)$ is precluded by the
strict precedence operator in the summation.

\Cref{hierrecall} can be alternately be viewed as a generalized area under the curve
(AUC) over the partial ranking induced by the hierarchical segmentation, where the
prediction threshold being varied is the depth within the estimated hierarchy $H_E$.
% Note that the precedence comparisons here are strict, so we never compare two $i$ and $j$ that merge simultaneously with $q$.
Aggregating over all frames $q$, we get the following average recall metric, which we
term the Structural Congruency Index (SCI):
\begin{equation}
SCI(H_E) \defeq \frac{1}{n} \sum_q g(q ; S_E)
\end{equation}

Note this loss is equivalent to that evaluated by for subjective artist similarity~\cite{mcfee2011}.
The difference here is that the reference rankings are induced from ordinal data, and not subject to consistency errors.

\subsection{Windowing in Time}

The calculation of SCI can be expensive ($\Oh\left(n^3\right)$ using a straightforward 
implementation).
The score also varies with $n$, so comparisons between tracks of differing length can
be probematic.  For extreme values of $n$, \cref{hierrecall} may be dominated by pairs
obviously irrelevant comparison points $j$ which lie far from $q$ in time.

Therefore, we propose to use a time window of $w$ seconds in order to both simplify the 
calculation of the metric and normalize its dynamic range.
We restrict the number of frames to consider to $m = \lceil w \cdot f_r \rceil$.
Adding this windowing property to the SIC equations yields:
\begin{equation}
  g(q, m ; H_E) \defeq \sum_{\substack{
  m_s \leq i,j \leq m_e \\ 
  H_R(q, i) \prec H_R(q, j) }} \frac{\ind{H_E(q, i) \prec H_E(q,
  j)}}{Z_q(m)},\label{windowrecall}
\end{equation}
\begin{equation}
SIC(H_E,m) \defeq \frac{1}{n} \sum_q g(q,m ; S_E),
\end{equation}
where $m_s = \min\{0,q-m/2\}$ and $m_e = \max\{q+m/2,n\}$ are the start and ending, respectively, of the current window in frame indices.

FIXME: Move this next part down to 3.1

Small windows (\eg, $w \leq 3$) only capture the local changes, 
losing precision since most of the segments will be more than this small window.
We want a window that is long enough to capture the boundaries of the current segments, and that is the same for all tracks in order to be able to easily compare results within a reasonable dynamic range.
In the task of music segmentation, most of the segments will be less than 45 seconds (30?).

\subsection{Transitivity}

Just as \cref{hierrecall} can be dominated by long-range interactions in the absence
of windowing, deep hierarchies can also pose a problem.  To see this, consider the 
sequence $H_R(q, i) \prec H_R(q, j) \prec H_R(q, k)$.  Due to the transitive
containment structure of $H_R$, we have $i \in H_R(q, i) \subseteq H_R(q, j)$.
Since the summation in \cref{hierrecall} ranges over all precedence comparisons, the
pair $(i, k)$ appears twice in the summation.  

To counteract this effect, the summation can be restricted to only range over direct
precedence relations.  In practice, this is accomplished by only comparing frames from
successive levels in the hierarchy.  As a result, redundant comparisons are
eliminated, and the dynamic range of $g$ is increased.

\section{Using SHAG}\label{sec:using_method}

In this section we discuss the behavior of SHAG by showing various synthetic examples, and compare them against other existing methods when possible.
We subdivide this section based on the type of annotations to be evaluated.

\subsection{Choosing a Time Window}

There's a time frame 

\subsection{Flat vs Flat}

\subsection{Flat vs Hierarchical}

\subsection{Hierarchical vs Hierarchical}

\section{Evaluating Automatic Algorithm}

Olda\cite{McFee2014} with SALAMI.


%\begin{table}
 %\begin{center}
 %\begin{tabular}{|l|l|}
  %\hline
  %String value & Numeric value \\
  %\hline
  %Hello ISMIR  & 2014 \\
  %\hline
 %\end{tabular}
%\end{center}
 %\caption{Table captions should be placed below the table.}
 %\label{tab:example}
%\end{table}

%\begin{figure}
 %\centerline{\framebox{
 %\includegraphics[width=\columnwidth]{figure.png}}}
 %\caption{Figure captions should be placed below the figure.}
 %\label{fig:example}
%\end{figure}

\section{Conclusions}

Our method will save us all.

\bibliography{refs}

%\bibliography{ismir2014template}

\end{document}
