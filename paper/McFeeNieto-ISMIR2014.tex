% -----------------------------------------------
% Template for ISMIR 2014
% (based on earlier ISMIR templates)
% -----------------------------------------------

\documentclass{article}
\usepackage{ismir2014,amsmath,cite}
\usepackage{graphicx}

% Title.
% ------
\title{Hierarchical Evaluation of Music Boundaries Using Rank Retrieval}

% Single address
% To use with only one author or several with the same address
% ---------------
%\oneauthor
% {Names should be omitted for double-blind reviewing}
% {Affiliations should be omitted for double-blind reviewing}

% Two addresses
% --------------
\twoauthors
  {First author} {School \\ Department}
  {Second author} {Company \\ Address}

% Three addresses
% --------------
%\threeauthors
  %{First author} {Affiliation1 \\ {\tt author1@ismir.edu}}
  %{Second author} {\bf Retain these fake authors in\\\bf submission to preserve the formatting}
  %{Third author} {Affiliation3 \\ {\tt author3@ismir.edu}}

% Four addresses
% --------------
%\fourauthors
%  {First author} {Affiliation1 \\ {\tt author1@ismir.edu}}
%  {Second author}{Affiliation2 \\ {\tt author2@ismir.edu}}
%  {Third author} {Affiliation3 \\ {\tt author3@ismir.edu}}
%  {Fourth author} {Affiliation4 \\ {\tt author4@ismir.edu}}

\begin{document}
%
\maketitle
%
\begin{abstract}
  Structure in music is traditionally analyzed in hierarchically: from the large scale sections in the highest level to the short melodic ideas in the motivic level. 
  Automatic approaches to estimate the structure of a given track typically output a flat estimation only, which is evaluated against a human annotated dataset, also annotated in a flat way.
  Recently, datasets composed of hierarchical evaluations have been published, but no methods are available in order to evaluate these richer annotations.
  We propose a method to evaluate hierarchical boundaries of music segments using rank retrieval techniques that can be used to evaluate flat annotations against hierarchical ones and vice versa.
  We show how our evaluation behaves with various synthetic and real world examples in order to illustrate its reliability and usability. 
\end{abstract}
%
\section{Introduction}\label{sec:introduction}

Segmentation is hierarchical.
SALAMI: dataset with hierarchical annotations.
Rank retrieval in order to evaluate hierarchical annotations.
Organization of the paper.

\section{Evaluation Description}\label{sec:eval_desc}

Description of the proposed evaluation.

\section{Using GAUC}\label{sec:using_method}

\subsection{Flat vs Flat}

\subsection{Flat vs Hierarchical}

\subsection{Hierarchical vs Hierarchical}

\section{Evaluating Automatic Algorithm}

Olda with SALAMI.


%\begin{table}
 %\begin{center}
 %\begin{tabular}{|l|l|}
  %\hline
  %String value & Numeric value \\
  %\hline
  %Hello ISMIR  & 2014 \\
  %\hline
 %\end{tabular}
%\end{center}
 %\caption{Table captions should be placed below the table.}
 %\label{tab:example}
%\end{table}

%\begin{figure}
 %\centerline{\framebox{
 %\includegraphics[width=\columnwidth]{figure.png}}}
 %\caption{Figure captions should be placed below the figure.}
 %\label{fig:example}
%\end{figure}

\section{Conclusions}

Our method will save us all.

\begin{thebibliography}{citations}

\bibitem {Author:00}
E. Author:
``The Title of the Conference Paper,''
{\it Proceedings of the International Symposium
on Music Information Retrieval}, pp.~000--111, 2000.

\bibitem{Someone:10}
A. Someone, B. Someone, and C. Someone:
``The Title of the Journal Paper,''
{\it Journal of New Music Research},
Vol.~A, No.~B, pp.~111--222, 2010.

\bibitem{Someone:04} X. Someone and Y. Someone: {\it Title of the Book},
    Editorial Acme, Porto, 2012.

\end{thebibliography}

%\bibliography{ismir2014template}

\end{document}
